\documentclass[a4paper, 10pt, french]{article}
% Préambule; packages qui peuvent être utiles
   \RequirePackage[T1]{fontenc}        % Ce package pourrit les pdf...
   \RequirePackage{babel,indentfirst}  % Pour les césures correctes,
                                       % et pour indenter au début de chaque paragraphe
   \RequirePackage[utf8]{inputenc}   % Pour pouvoir utiliser directement les accents
                                     % et autres caractères français
   \RequirePackage{lmodern,tgpagella} % Police de caractères
   \textwidth 17cm \textheight 25cm \oddsidemargin -0.24cm % Définition taille de la page
   \evensidemargin -1.24cm \topskip 0cm \headheight -1.5cm % Définition des marges
   \RequirePackage{latexsym}                  % Symboles
   \RequirePackage{amsmath}                   % Symboles mathématiques
   \RequirePackage{tikz}   % Pour faire des schémas
   \RequirePackage{graphicx} % Pour inclure des images
   \RequirePackage{listings} % pour mettre des listings

   \newcommand{\set}[1]{\left\{ #1 \right\}}
   \newcommand{\abs}[1]{\left| #1 \right|}
% Fin Préambule; package qui peuvent être utiles

\title{Rapport de TP 4MMAOD : Génération de patch optimal}
\author{
Matthieu Bergeron (ISI2A\_G2)
Tom Cornebize (ISI2A\_G2)
}

\setlength\parindent{0pt}

\begin{document}

\maketitle

%%%%%%%%%%%%%%%%%%%%%%%%%%%%%%%%%%%%%%%%%%%%%%
\section{Principe de notre  programme (1 point)}

\subsection{Équation de Bellman}

Nous rappelons l'équation de Bellman sur laquelle est basé notre programme.

Soient $\ell_1$ (resp. $\ell_2$) le nombre de lignes du fichier $F_1$ (resp. $F_2$).

Nous noterons $F_i[n]$, avec $n \in [1, \ell_i]$, pour dénoter la $n$-ème ligne du fichier
$F_i$. Nous noterons $\abs{F_i[n]}$ la longueur de cette ligne.

Pour $n_1 \in [1, \ell_1]$ et $n_2 \in [1, \ell_2]$, nous définissons $cout(n_1, n_2)$
comme étant le coût du patch de coût minimal permettant de transformer les $n_1$
premières lignes du fichier $F_1$ en les $n_2$ premières lignes du fichier $F_2$.

La fonction $cout$ peut s'écrire de manière inductive :

\[
    cout(n_1, n_2) = \min
    \begin{cases}
        cout(n_1-1, n_2-1)                          & \text{si } F_1[n_1] = F_2[n_2]\\
        cout(n_1-1, n_2-1) + 10 + \abs{F_2[n_2]}    & \text{(substitution)}\\
        cout(n_1-1, n_2) + 10                       & \text{(destruction)}\\
        \min_{i \in [2, n_1]} cout(n_1-i, n_2) + 15          & \text{(destruction multiple)}\\
        cout(n_1, n_2-1) + 10 + \abs{F_2[n_2]}      & \text{(ajout)}
    \end{cases}
\]

Avec comme condition initiale $cout(0, 0) = 0$, $cout(1, 0) = 10$, $cout(n_1, 0) = 15$
si $n_1 > 1$ et $cout(0, n_2)$ = $10n_2$.

\subsection{Implémentation}

Notre programme est une implémentation itérative de cette équation. Il est constitué
de deux boucles \emph{for} imbriquées. La boucle extérieure itère sur l'indice $n_2$
du fichier de sortie et la boucle intérieure itère sur l'indice $n_1$ du fichier
d'entrée.

Notons que le calcul de $cout(n_1, n_2)$ dépend de $cout(n_1-1, n_2)$,
$cout(n_1, n_2-1)$, $cout(n_1-1, n_2-1)$ et $cout(n_1-i, n_2)$ pour tout $i$.
Nous n'avons donc pas besoin de garder en mémoire les valeurs de $cout(i, j)$ pour
$j < n_2-1$.

Nous manipulons deux tableaux de $n_1+1$ cases. Le premier représente les solutions
de $cout(i, n_2)$ pour tout $i$ et pour le $n_2$ courant, le second représente
les solutions de $cout(i, n_2-1)$ pour tout $i$ et pour le $n_2$ courant.
Ainsi, nous évitions de devoir utiliser une matrice de $(n_1+1)\times(n_2+2)$ cases
et faisons donc une économie de mémoire importante.

Pour la destruction multiple, nous devons calculer le patch de coût minimal parmi
tous les patchs précédemment calculés pour une même valeur de $n_2$.
Plutôt que de reparcourir le tableau à chaque fois, nous stockons l'indice de
ce minimum, que nous mettons à jour à chaque nouveau patch calculé.

%%%%%%%%%%%%%%%%%%%%%%%%%%%%%%%%%%%%%%%%%%%%%%
\section{Analyse du coût théorique (3 points)}
{\em Donner ici l'analyse du coût théorique de votre programme en fonction des nombres $n_1$ et $n_2$ de lignes
et $c_1$ et $c_2$ de caractères des deux fichiers en entrée.
 Pour chaque coût, donner la formule qui le caractérise (en justifiant brièvement pourquoi cette formule correspond à votre programme),
 puis l'ordre du coût en fonction de $n_1, n_2, c_1, c_2$ en notation $\Theta$ de préférence, sinon $O$.}

  \subsection{Nombre  d'opérations en pire cas\,: }
    \paragraph{Justification\,: }
    {\em La justification peut être par exemple de la forme: \\
       "Le programme itératif contient les boucles $k_1=...$, $k_2= ...$ etc correspondant à la somme
      $$T(n_1, n_2, c_1, c_2) = \sum_{k_1=...}^{...} ... \sum ... + \sum_{i=...}^{...} ...$$
      somme que nous avons calculée (ou majorée) par la technique de  ... " \\
      ou  encore\,:  \\
      "les appels récursifs du programme permettent de modéliser son coût par le système d'équations aux récurrences
      $$T(k_1, k_2) = ...  \mbox{~avec~les~conditions~initiales~....~} $$
      Le coût indiqué est obtenu en résolvant ce système par la méthode de  .... "
    }
  \subsection{Place mémoire requise\,: }
    \paragraph{Justification\,: }

  \subsection{Nombre de défauts de cache sur le modèle CO\,: }
    \paragraph{Justification\,: }


%%%%%%%%%%%%%%%%%%%%%%%%%%%%%%%%%%%%%%%%%%%%%%
\section{Compte rendu d'expérimentation (2 points)}
  \subsection{Conditions expérimentaless}
     {\em Décrire les conditions permettant la reproductibilité des mesures: on demande la description
      de la machine et la méthode utilisée pour mesurer le temps.
     }

    \subsubsection{Description synthétique de la machine\,:}
      {\em indiquer ici le  processeur et sa fréquence, la mémoire, le système d'exploitation.
       Préciser aussi si la machine était monopolisée pour un test, ou notamment si
       d'autres processus ou utilisateurs étaient en cours d'exécution.
      }

    \subsubsection{Méthode utilisée pour les mesures de temps\,: }
      {\em préciser ici  comment les mesures de temps ont été effectuées (fonction appelée) et l'unité de temps; en particulier,
       préciser comment les 5 exécutions pour chaque test ont été faites (par exemple si le même test est fait 5 fois de suite, ou si les tests sont alternés entre
       les mesures, ou exécutés en concurrence etc).
      }

  \subsection{Mesures expérimentales}
    {\em Compléter le tableau suivant par les temps d'exécution mesurés pour chacun des 6 benchmarks imposés
              (temps minimum, maximum et moyen sur 5 exécutions)
    }

    \begin{figure}[h]
      \begin{center}
        \begin{tabular}{|l||r||r|r|r||}
          \hline
          \hline
            & coût         & temps     & temps   & temps \\
            & du patch     & min       & max     & moyen \\
          \hline
          \hline
            benchmark1 &      &     &     &     \\
          \hline
            benchmark2 &      &     &     &     \\
          \hline
            benchmark3 &      &     &     &     \\
          \hline
            benchmark4 &      &     &     &     \\
          \hline
            benchmark5 &      &     &     &     \\
          \hline
            benchmark6 &      &     &     &     \\
          \hline
          \hline
        \end{tabular}
        \caption{Mesures des temps minimum, maximum et moyen de 5 exécutions pour les 6 benchmarks.}
        \label{table-temps}
      \end{center}
    \end{figure}

\subsection{Analyse des résultats expérimentaux}
{\em Donner  une réponse justifiée  à la question\,:
              les  temps mesurés correspondent ils  à votre analyse théorique (nombre d’opérations et défauts de cache) ?
}

%%%%%%%%%%%%%%%%%%%%%%%%%%%%%%%%%%%%%%%%%%%%%%
\section{Question\,: et  si le coût d'un patch était sa taille en octets ? (1 point)}
{\em Préciser le principe de la résolution choisie (parmi celles vues en cours); donner  les modifications à apporter (soit à vos  équations, soit à votre programme, au choix)
pour s'adapter à cette nouvelle fonction de coût.
}

\end{document}
%% Fin mise au format
