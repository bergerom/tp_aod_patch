\documentclass[a4paper, 10pt, french]{article}
% Préambule; packages qui peuvent être utiles
   \RequirePackage[T1]{fontenc}        % Ce package pourrit les pdf...
   \RequirePackage{babel,indentfirst}  % Pour les césures correctes,
                                       % et pour indenter au début de chaque paragraphe
   \RequirePackage[utf8]{inputenc}   % Pour pouvoir utiliser directement les accents
                                     % et autres caractères français
   \RequirePackage{lmodern,tgpagella} % Police de caractères
   \textwidth 17cm \textheight 25cm \oddsidemargin -0.24cm % Définition taille de la page
   \evensidemargin -1.24cm \topskip 0cm \headheight -1.5cm % Définition des marges
   \RequirePackage{latexsym}                  % Symboles
   \RequirePackage{amsmath}                   % Symboles mathématiques
   \RequirePackage{tikz}   % Pour faire des schémas
   \RequirePackage{graphicx} % Pour inclure des images
   \RequirePackage{listings} % pour mettre des listings

   \newcommand{\set}[1]{\left\{ #1 \right\}}
   \newcommand{\abs}[1]{\left| #1 \right|}
% Fin Préambule; package qui peuvent être utiles
\usepackage{moreverb}
\title{Rapport de TP 4MMAOD : Génération de patch optimal}
\author{
Matthieu Bergeron (ISI2A\_G2)
Tom Cornebize (ISI2A\_G2)
}

\setlength\parindent{0pt}

\begin{document}

\maketitle

%%%%%%%%%%%%%%%%%%%%%%%%%%%%%%%%%%%%%%%%%%%%%%
\section{Principe de notre  programme (1 point)}

\subsection{Équation de Bellman}

Nous rappelons l'équation de Bellman sur laquelle est basé notre programme.

Soient $\ell_1$ (resp. $\ell_2$) le nombre de lignes du fichier $F_1$ (resp. $F_2$).

Nous noterons $F_i[n]$, avec $n \in [1, \ell_i]$, pour dénoter la $n$-ème ligne du fichier
$F_i$. Nous noterons $\abs{F_i[n]}$ la longueur de cette ligne.

Pour $n_1 \in [1, \ell_1]$ et $n_2 \in [1, \ell_2]$, nous définissons $cout(n_1, n_2)$
comme étant le coût du patch de coût minimal permettant de transformer les $n_1$
premières lignes du fichier $F_1$ en les $n_2$ premières lignes du fichier $F_2$.

La fonction $cout$ peut s'écrire de manière inductive :

\[
    cout(n_1, n_2) = \min
    \begin{cases}
        cout(n_1-1, n_2-1)                          & \text{si } F_1[n_1] = F_2[n_2]\\
        cout(n_1-1, n_2-1) + 10 + \abs{F_2[n_2]}    & \text{(substitution)}\\
        cout(n_1-1, n_2) + 10                       & \text{(destruction)}\\
        \min_{i \in [2, n_1]} cout(n_1-i, n_2) + 15          & \text{(destruction multiple)}\\
        cout(n_1, n_2-1) + 10 + \abs{F_2[n_2]}      & \text{(ajout)}
    \end{cases}
\]

Avec comme condition initiale $cout(0, 0) = 0$, $cout(1, 0) = 10$, $cout(n_1, 0) = 15$
si $n_1 > 1$ et $cout(0, n_2)$ = $10n_2$.

\subsection{Implémentation}

Notre programme est une implémentation itérative de cette équation. Il est constitué
de deux boucles \emph{for} imbriquées. La boucle extérieure itère sur l'indice $n_2$
du fichier de sortie et la boucle intérieure itère sur l'indice $n_1$ du fichier
d'entrée.
Voici le pseudo-code de cet algorithme :
\begin{verbatimtab}
	Retourner la réponse directement si on est dans un cas trivial
	Faire les intialisations nécéssaires
	for(i=1..n2)
	  for(j=1..n1)
		Calcul du patch minimum (i,j) -- Avec l'équation de Bellman
	
	Retourner le dernier résultat du tableau
\end{verbatimtab}
Notons que le calcul de $cout(n_1, n_2)$ dépend de $cout(n_1-1, n_2)$,
$cout(n_1, n_2-1)$, $cout(n_1-1, n_2-1)$ et $cout(n_1-i, n_2)$ pour tout $i$.
Nous n'avons donc pas besoin de garder en mémoire les valeurs de $cout(i, j)$ pour
$j < n_2-1$.

Nous manipulons deux tableaux de $n_1+1$ cases. Le premier représente les solutions
de $cout(i, n_2)$ pour tout $i$ et pour le $n_2$ courant, le second représente
les solutions de $cout(i, n_2-1)$ pour tout $i$ et pour le $n_2$ courant.
Ainsi, nous évitions de devoir utiliser une matrice de $(n_1+1)\times(n_2+2)$ cases
et faisons donc une économie de mémoire importante.

Pour la destruction multiple, nous devons calculer le patch de coût minimal parmi
tous les patchs précédemment calculés pour une même valeur de $n_2$.
Plutôt que de reparcourir le tableau à chaque fois, nous stockons l'indice de
ce minimum, que nous mettons à jour à chaque nouveau patch calculé.

%%%%%%%%%%%%%%%%%%%%%%%%%%%%%%%%%%%%%%%%%%%%%%
\section{Analyse du coût théorique (3 points)}

  \subsection{Nombre  d'opérations en pire cas\,:}
    En reprenant les notations de la partie précédente : \\ \\
    $$T(n1,n2,c1,c2) = \underbrace{Coût(\text{Cas triviaux})}_{O(1)} + 
    \underbrace{Coût(\text{Intialisations})}_{O(n1)} + 
    \sum_{i=1}^{n2} \sum_{i=1}^{n1} Coût((\text{Calcul patch (i,j)})$$ \\
	On néglige donc les deux premiers coûts, ce qui nous donne : \\ \\
	$$T(n1,n2,c1,c2) = n1 \times n2 \times Coût(\text{Calcul patch (i,j)})$$
	\\ 
	Nous devons donc calculer le coût du calcul du patch (i,j), c'est le coût de la fonction \emph{compute\_at\_indexes(index_in,index_out)}.\\ \\
	Décomposons les coûts internes à cette fonction dans le pire cas :
	\begin{itemize}
	\item Une comparaison de chaine de caractère $\Rightarrow  O(\max(li,lj))$ en notant li (\textit{resp lj}) la longueur de la ligne de i (\textit{resp j}).
	\item Un cout constant pour le reste des opérations
	\end{itemize}
	~\\
	On en déduit que le cout au pire cas de cette fonction vaut : $\max(li,lj)$ (les autres coûts étant constants)
	\\ 
	$$T(n1,n2,c1,c2) = \sum_{i=1}^{n2} \sum_{i=1}^{n1} \max(li,lj) \leq n1 \times n2 \times L \ \ \text{avec } L = \max(c1,c2)$$
	\\
	 \ \ \ $\text{Finalement, } T(n1,n2,c1,c2) = O(n1 \times n2 \times L)\ \ \text{avec } L = \max(c1,c2) $
  \subsection{Place mémoire requise\,: }
	Nous utilisons deux listes pous stocker les informations nécéssaire au calcul du coût 		  courant : une pour la ligne courante, l'autre pour la ligne
    précédente. Le patch de cout minimum sur une ligne est stocké avec son indice.\\ \\
    Le cout total en mémoire doit donc prendre en compte : 
    \begin{itemize}
    \item Le coût du stockage des fichiers en entrée : \ $O(c1 + c2)$
    \item Le coût des mémorisations :  $\underbrace{n1}_{ligne\ courante} + \underbrace{n1}_{ligne\ précédente} + \underbrace{1}_{indice \ minimum} = O(n1)$
    \end{itemize}
    ~\\
    Ce qui donne un coût de $\fbox{O(c1 + c2 + n1)}$
  \subsection{Nombre de défauts de cache sur le modèle CO\,: }
Pour simplifier les choses, nous pouvons considérer que nos valeurs sont stockées dans un grand tableau, de taille $2 \times n1 + 1$.\\
    A l'itération i, le programme a besoin de quatres cases en mémoire :
    \begin{itemize}
    \item $(i-1)$ pour la destruction
    \item $(i - i_{min})$ pour la multi-destruction, $i_{min}$ étant défini tel que $T[i-i_{min}]$ est le patch de cout minimal.
    \item $(i - n1)$ pour l'addition
    \item $(i - n1 - 1)$ pour la substitution et l'identité \\
    \end{itemize}
    \ \ \ \ \ \ \ \ 
    \begin{tabular}{|c|c|c|c|c|c|c|c|c|c|}
    	\hline 
    	
    	 $i - n1 - 1$ & $i - n1$  &  .. & .. & $i - i_{min}$ & .. & .. & $i - 1$ & $i$ \\
    	 
    	\hline
    \end{tabular}
	 \\ \\
	Le nombre de défauts de cache dépend donc de la taille totale Z du cache :\\ \\
	\textbf{Si $Z > n1$ :} \\
		 - Un défaut de cache inévitable pour écrire les tranches du tableau dans le cache : $\theta(\dfrac{n1}{L})$ \\
		 - Pas de défaut de chache pour accéder aux cases mémoires. \\
		 $\Rightarrow \theta(\dfrac{n1}{L})$ \\ \\ 
	\textbf{Si $Z < n1$ :} \\
	- Défaut inévitable : $\theta(\dfrac{n1}{L})$ \\
	- Défaut à chaque calcul pour retrouver les patchs dans les cases $(i - n1)$ et  $(i - n1 - 1)$ : $\theta(n1)$ \\
 	$\Rightarrow \theta(n1 (1 + \dfrac{1}{L}) )$ \\ \\
 	\\
 	Pour simplifier les calculs, nous avons considéré que l'indice $i_{min}$ est soit proche de $i - n1$, soit proche de $i$. \\
 	Avec cette simplification, nous sommes sûr d'avoir le patch de cout minimum sur la ligne dans le cache. En effet si la case du patch dans
 	le tableau est proche
 	de la case courante,  elle est déjà dans le cache.Sinon, elle est récupérée avec le cache-miss provoqué par l'accès à l'indice
 	$i - n1$.

%%%%%%%%%%%%%%%%%%%%%%%%%%%%%%%%%%%%%%%%%%%%%%
\section{Compte rendu d'expérimentation (2 points)}
  \subsection{Conditions expérimentaless}
     {\em Décrire les conditions permettant la reproductibilité des mesures: on demande la description
      de la machine et la méthode utilisée pour mesurer le temps.
     }

    \subsubsection{Description synthétique de la machine\,:}
      {\em indiquer ici le  processeur et sa fréquence, la mémoire, le système d'exploitation.
       Préciser aussi si la machine était monopolisée pour un test, ou notamment si
       d'autres processus ou utilisateurs étaient en cours d'exécution.
      }

    \subsubsection{Méthode utilisée pour les mesures de temps\,: }
      {\em préciser ici  comment les mesures de temps ont été effectuées (fonction appelée) et l'unité de temps; en particulier,
       préciser comment les 5 exécutions pour chaque test ont été faites (par exemple si le même test est fait 5 fois de suite, ou si les tests sont alternés entre
       les mesures, ou exécutés en concurrence etc).
      }

  \subsection{Mesures expérimentales}
    {\em Compléter le tableau suivant par les temps d'exécution mesurés pour chacun des 6 benchmarks imposés
              (temps minimum, maximum et moyen sur 5 exécutions)
    }

    \begin{figure}[h]
      \begin{center}
        \begin{tabular}{|l||r||r|r|r||}
          \hline
          \hline
            & coût         & temps     & temps   & temps \\
            & du patch     & min       & max     & moyen \\
          \hline
          \hline
            benchmark1 &      &     &     &     \\
          \hline
            benchmark2 &      &     &     &     \\
          \hline
            benchmark3 &      &     &     &     \\
          \hline
            benchmark4 &      &     &     &     \\
          \hline
            benchmark5 &      &     &     &     \\
          \hline
            benchmark6 &      &     &     &     \\
          \hline
          \hline
        \end{tabular}
        \caption{Mesures des temps minimum, maximum et moyen de 5 exécutions pour les 6 benchmarks.}
        \label{table-temps}
      \end{center}
    \end{figure}

\subsection{Analyse des résultats expérimentaux}
{\em Donner  une réponse justifiée  à la question\,:
              les  temps mesurés correspondent ils  à votre analyse théorique (nombre d’opérations et défauts de cache) ?
}

%%%%%%%%%%%%%%%%%%%%%%%%%%%%%%%%%%%%%%%%%%%%%%
\section{Question\,: et  si le coût d'un patch était sa taille en octets ? (1 point)}
{\em Préciser le principe de la résolution choisie (parmi celles vues en cours); donner  les modifications à apporter (soit à vos  équations, soit à votre programme, au choix)
pour s'adapter à cette nouvelle fonction de coût.
}

\end{document}
%% Fin mise au format
